% stephane [DOT] adjemian [AT] univ [DASH] lemans [DOT] fr
\documentclass[10pt,a4paper,notitlepage,twocolumn]{article}
\usepackage{amsmath}
\usepackage{amssymb}
\usepackage{amsbsy}
\usepackage[T1]{fontenc}
\usepackage[utf8x]{inputenc}
\usepackage{palatino}
\usepackage{scrtime}
\usepackage[frenchb]{babel}
\usepackage{float}

%%%%%%%%%%%%%%%%%%%%%%%%%%%%%%%%%%%%%%%%%%%%%%%%%%%%%%%%%%%%%%%%%%%%%%%%%%%%%%%%%%%%%%%%%%%%%%%%%%%%

\newcommand{\exercice}[1]{\textsc{\textbf{Exercice}} #1}
\newcommand{\question}[1]{\textbf{(#1)}}
\setlength{\parindent}{0cm}



\begin{document}


\title{\textsc{Économétrie des Variables Qualitatives}}
\author{\textsc{Université du Mans (Examen, L3)}}
\date{}


\maketitle

\thispagestyle{empty}

\textbf{\textsc{Exercice} I}\newline

\question{1} Soit un échantillon de variables
dichotomiques $(y_1, \dots, y_N)$ identiquement et indépendament
distribués selon une loi de Bernouilli de paramètre $p$. Chaque $y$
vaut 1 avec probabilité $p$ ou 0 avec probabilité $(1-p).$ Calculer
l'espérance et la variance de $y$.\newline

\question{2} Écrire la fonction de probabilité de $y$.\newline

\question{3} Écrire la fonction de log-vraisemblance.\newline

\question{4} Calculer l'estimateur du maximum de vraisemblance
de $p$.\newline

\question{5} Calculer la variance de cet estimateur et caractériser le
comportement asymptotique de cet estimateur.\newline

\bigskip

\textbf{\textsc{Exercice} II}\newline

On s'intéresse aux conséquences de l'omission d'une variable exogène
dans un modèle logit. On suppose que les données sont générées par un
modèle logit admettant deux variables explicatives, $x_1$ et $x_2$. La
probabilité que $y_i$ ($i=1,\dots,N$) soit égal à 1 est~:
\[
\mathbb P \left( y_i=1 \right) = F\left( x_{1,i}b_1 + x_{2,i}b_2\right)
\]
où $F$ est la fonction de répartition d'une loi logistique, $b_1$
et $b_2$ les vraies valeurs (inconnues) des paramètres.\newline

\question{1} Donner une expression de la fonction de répartition d'une
loi logistique, $F(z)$.\newline

\question{2} Donner l'expression de la fonction de densité de
probabilité d'une loi logistique, puis montrer que~:
\[
  \frac{\mathrm d}{\mathrm d z}\log F(z) = 1-F(z)
\]
et
\[
  \frac{\mathrm d}{\mathrm d z}\log 1-F(z) = -F(z)
\]

\question{3} L'économètre estime le modèle logit en omettant la
variable $x_2$ (qu'il n'observe pas). Écrire la fonction de
vraisemblance.\newline

\question{4} Calculer la condition du premier ordre, définissant
l'estimateur du maximum de vraisemblance du paramètre $b_1$. Montrer
que cette condition peut s'écrire sous la forme~:
\[
\sum_{i=1}^N \left( y_i - F\left( x_{1,i}\hat b_1 \right) \right)x_{1,i} = 0
\]
Interpréter cette condition, en la comparant à la condition
d'identification dans modèle linéaire (avec les MCO par
exemple). Est-il possible d'obtenir une expression analytique
de $\hat b_1$~? Pourquoi~? Expliquer comment on peut obtenir $\hat b_1$.\newline

\question{5} Expliquer pourquoi la condition du premier ordre est liée à la condition moment suivante~:
\[
\mathbb E\left[ \left( y-F(x_1\hat b_1) \right)x_1 \right] = 0
\]

\question{6} En utilisant la loi des espérances itérées et le modèle
générateur des données, montrer que l'on peut réécrire la condition de
moment sous la forme~:
\[
\mathbb E_x\left[  \left( F(x_1b_1 + x_2b_2) - F(x_1\hat b_1) \right)x_1\right] = 0
\]

\question{7} Sous quelle condition $\hat b_1$ est-il un estimateur
convergent~? Comparer avec la condition d'absence de biais de variable
omise dans un modèle linéaire.

\end{document}
