% stephane [DOT] adjemian [AT] univ [DASH] lemans [DOT] fr
\documentclass[10pt,a4paper,notitlepage,twocolumn]{article}
\usepackage{amsmath}
\usepackage{amssymb}
\usepackage{amsbsy}
\usepackage[T1]{fontenc}
\usepackage[utf8x]{inputenc}
\usepackage{palatino}
\usepackage{scrtime}
\usepackage[french]{babel}
\usepackage{float}

%%%%%%%%%%%%%%%%%%%%%%%%%%%%%%%%%%%%%%%%%%%%%%%%%%%%%%%%%%%%%%%%%%%%%%%%%%%%%%%%%%%%%%%%%%%%%%%%%%%%

\newcommand{\exercice}[1]{\textsc{\textbf{Exercice}} #1}
\newcommand{\question}[1]{\textbf{(#1)}}
\setlength{\parindent}{0cm}



\begin{document}


\title{\textsc{Économétrie des Variables Qualitatives}}
\author{\textsc{Université du Mans (Rattrapage, L3)}}
\date{Mardi 17 juin 2025}


\maketitle

\question{1} Soit un échantillon de variables
dichotomiques $(y_1, \dots, y_N)$ identiquement et indépendament
distribués selon une loi de Bernouilli de paramètre $p$. Chaque $y$
vaut 1 avec probabilité $p$ ou 0 avec probabilité $(1-p).$ Calculer
l'espérance et la variance de $y$.\newline

\question{2} Écrire la fonction de probabilité de $y$.\newline

\question{3} Écrire la fonction de log-vraisemblance.\newline

\question{4} Calculer l'estimateur du maximum de vraisemblance
de $p$.\newline

\question{5} Calculer la variance de cet estimateur (à partir de
l'expression de l'estimateur), caractériser le comportement
asymptotique de cet estimateur. Déterminer une fonction $f(N)$ telle
que $f(N)\hat p$ converge vers une loi normale (justifier votre
réponse).\newline

\question{6} Nous supposons maintenant que la probabilité est
spécifique à chaque $y_i$, les variables aléatoires ne sont plus
identiquement et indépendament distribuées, seulement indépendament
distribuées. Chaque probabilité $p_i$ est déterminée par un ensemble
de variables exogènes et de paramètres~:
\[
  p_i = p(X_i, \beta)
\]
On cherche maintenant à estimer le vecteur de paramètres $\beta$. Écrire la log-vraisemblance.\newline

\question{7} Calculer le score montrer qu'il est d'espérance nulle.\newline

\question{8} Calculer la matrice hessienne de la
log-vraisemblance. Donner une expression simple de son espérance en
exploitant les propriétés de la loi de Bernouilli. Dans quel cadre
cette expression peut-elle être utilisée~?\newline

\question{9} Afin d'expliciter la fonction $p$ qui détermine la
probabilité que $y_i$ soit égal à 1, on postule un  modèle à variable
latente. On définit la variable latente $z_i$ par
\[
 z_i = X_i\beta + u_i
\]
où $u_i$ est une variable aléatoire identiquement et indépendament
distribuée d'espérance nulle, $X_i$ est un vecteur $1\times K$ de
variables exogènes et $\beta$ un vecteur $K\times 1$ de paramètres. On
note $F$ la fonction de répartition de $u_i$. On postule alors~:
\[
  y_i = \begin{cases}
    & 1 \quad\text{si }z_i> 0\\
    & 0 \quad\text{sinon}
  \end{cases}
\]
Déterminer la forme de la fonction $p(X_i, \beta)$.\newline

\question{10} Dans la suite on suppose que $u_i$ suit une loi
logistique. Écrire la log-vraisemblance du modèle, en exploitant les
propriétés de la loi logistique pour simplifier son
expression.\newline

\question{11} Calculer la matrice hessienne et montrer que celle-ci
est définie négative. Quelles sont les conséquences~?\newline

\question{12} Comment estimer le vecteur de paramètre $\beta$~?

\end{document}
